\documentclass[7pt]{scrartcl}
\usepackage{multicol}
\usepackage{calc}
\usepackage{ifthen}
\usepackage[landscape]{geometry}
\usepackage{xcolor}
\usepackage{colortbl}
\usepackage{tabularx}
\usepackage{multirow}
\usepackage{graphicx}

\ifthenelse{\lengthtest { \paperwidth = 11in}}
	{ \geometry{top=.5in,left=.5in,right=.5in,bottom=.5in} }
	{\ifthenelse{ \lengthtest{ \paperwidth = 297mm}}
		{\geometry{top=1cm,left=1cm,right=1cm,bottom=1cm} }
		{\geometry{top=1cm,left=1cm,right=1cm,bottom=1cm} }
	}
	
\pagestyle{empty}

\makeatletter
\renewcommand{\section}{\@startsection{section}{1}{0mm}%
                                {-2ex plus -.5ex minus -.2ex}%
                                {0.5ex plus .2ex}%x
                                {\normalfont\large\bfseries}}
\makeatother

\def\BibTeX{{\rm B\kern-.05em{\sc i\kern-.025em b}\kern-.08em
    T\kern-.1667em\lower.7ex\hbox{E}\kern-.125emX}}

% Don't print section numbers
\setcounter{secnumdepth}{0}


\setlength{\parindent}{0pt}
\setlength{\parskip}{0pt plus 0.5ex}

\setlength{\premulticols}{0pt}
\setlength{\postmulticols}{0pt}
\setlength{\multicolsep}{0pt}
\setlength{\columnsep}{0pt}

\raggedright
\footnotesize

\definecolor{Gray}{gray}{0.85}
\definecolor{Red}{rgb}{1.0 0.7 0.7}
\definecolor{Green}{rgb}{0.7 1.0 0.7}
\newcolumntype{g}{>{\columncolor{Gray}}l}
\newcolumntype{G}{>{\columncolor{Gray}}X}
\newcolumntype{a}{>{\columncolor{Red}}l}
\newcolumntype{A}{>{\columncolor{Red}}X}
\newcolumntype{i}{>{\columncolor{Green}}l}
\newcolumntype{I}{>{\columncolor{Green}}X}

\newcommand{\mytable}[2]{\begin{tabularx}{0.33\textwidth}{|#1|}\hline #2 \hline\end{tabularx}}
\newcommand{\specialcell}[2][l]{\begin{tabular}[#1]{@{}l@{}}#2\end{tabular}}
\newcommand{\sct}[1]{\specialcell[t]{#1}}
%\newcommand*{\thead}[1]{\multicolumn{1}{c}{\bfseries #1}}
\newcommand{\thead}[1]{\textbf{#1}}

\newcommand{\myopt}[1]{\hspace{3ex}\myoptc{#1}}
\newcommand{\myoptc}[1]{\texttt{#1}}
\newcommand{\myoptd}[1]{\hspace{3ex}\fontsize{6pt}{6pt}\selectfont #1}
\newcommand{\code}[1]{\texttt{\textbf{#1}}}

\newcommand{\attention}[1]{\multicolumn{1}{|a}{#1}}
\newcommand{\important}[1]{\multicolumn{1}{|i}{#1}}

\newcommand{\repo}[0]{\textit{repo}}
\newcommand{\commit}[0]{\textit{commit}}
\newcommand{\branch}[0]{\textit{branch}}
\newcommand{\rbranch}[0]{\textit{rbranch}}
\newcommand{\file}[0]{\textit{file}}
\newcommand{\dir}[0]{\textit{dir}}
\newcommand{\tag}[0]{\textit{tag}}
\newcommand{\remote}[0]{\textit{remote}}
\newcommand{\url}[0]{\textit{url}}

\begin{document}

\begin{multicols*}{3}

\begin{center}
	\LARGE{\textbf{Bash Programming Cheat Sheet}}\\
	\large{\reflectbox{\textcopyright} Erik E. Lorenz, \today}
\end{center}

\section{Internal Files and Directories}
\mytable{gX}{
	\code{$\sim$/.bashrc}	&	user-specific global functions and aliases	\\
	\code{$\sim$/.bash\_profile}	&	similar to \code{$\sim$/.bashrc} \\
	\code{$\sim$/.bash\_history}	&	list of previous bash commands \\
	\code{$\sim$/.bash\_logout}	&	runs on bash logout	\\
	\code{/bin/bash}	&	location of the \code{bash} executable	\\
}

\section{Terms}
\mytable{iXi}{
	\thead{term}	&	\thead{description}	&	\thead{examples}	\\
	\user{}	&	a user account of the system	&	\sct{root\\e.lorenz}\\
	\file{}	&	regular file &	\sct{$\sim$/file.txt\\code/asd/src/main.cpp}	\\
	\dir{}	&	regular directory &	\sct{$\sim$/directory\\/etc}	\\
	\cmd{}	&	any command	&	\sct{\code{echo}\\\code{date +\%F}}	\\
	\host{}	&	name or ip of a remote machine	&	\sct{enssim.etit.tu-chemnitz.de\\134.109.52.89}	\\
	\port{}	&	a network port for communication with a program	&	\sct{22\\31159}	\\
	\url{}	&	uniform resource locator	&	\sct{http://\host{}:port/\dir{}/\file{}}	\\
	\pid{}	&	process id	&	18738	\\
	alias	&	command alias	& \code{alias ssk='ssh enssim'}	\\
	export	&	define an environment variable	&	\code{export PATH=$\sim$/bin:\$PATH}	\\
	source	&	run a script that sets environment variables/aliases	&	\sct{\code{. $\sim$/.bashrc}\\\code{source $\sim$/.bashrc}}	\\
}

\section{Useful Environment Variables}
\mytable{iX} {
	\code{\$HOME}	&	home directory. Usually \code{/home/\user{}}	\\
	\code{$\sim$}	&	same as \code{\$HOME}	\\
	\code{\$USER}	&	name of the current user	\\
	\code{\$UID, \$EUID}	&	user id, effective user id	\\
	\code{\$PATH}	&	colon-separated list of search directories for binaries	\\
	\code{\$LIBRARY\_PATH}	&	search paths for .so and .a files at compile time	\\
	\code{\$LD\_LIBRARY\_PATH}	&	search paths for .so and .a files at run time	\\
	\code{\$PWD}	&	current working directory	\\
	\code{\$EDITOR}	&	preferred command line text editor, e.g. \code{vim}	\\
	\code{\$IFS}	&	internal field separator, e.g. for \code{for...in} constructs	\\
	\code{\$LINENO}	&	current line number in a script, e.g. for debugging	\\
	\code{\$COLUMNS}	&	width of the terminal	\\
	\code{\$LINES}	&	height of the terminal	\\
	\code{\$LANG}	&	preferred language of the user	\\
	\code{\$SHELL}	&	path of the shell-executable. Should be \code{/bin/bash}	\\
	\code{\$SHLVL}	&	shell nesting level on the current machine	\\
	\code{\$\$}	&	\pid{} of the current script or bash instance	\\
	\code{\$PPID}	&	\pid{} of the parent process	\\
	\code{\$!}	&	\pid{} of the last child process (see \code{Forking})	\\
	\code{\$0}	&	command used to run this script or bash instance	\\
	\code{\$@}	&	array of arguments of a script or function	\\
	\code{\$1, \$2, ... \$9}	&	first, second, ... ninth argument	\\
	\code{!!}	&	previous command	\\
	\code{!\$}	&	last argument of the previous command	\\
	\code{!\^}	&	first arguments of the previous command	\\
	\code{!:1, !:2, ...}	&	arguments of the last command	\\
	\code{!:1-}	&	all arguments of the last command	\\
}

\section{Cheat Sheet Color Coding}
\mytable{gX}{
	\important{\code{cmd}}	&	Most frequent commands	\\
	\code{cmd}	&	Usually not harmful	\\
	\attention{\code{cmd}}	&	deletes data, requires root or bad programming	\\
}

\section{Debugging}
\mytable{gX}{
	\code{set -x}	&	print every command before execution	\\
	\code{trap read debug}	&	confirm every command with [Enter]	\\
}

\section{Hotkeys}
\mytable{gX}{
	\important{Tab}	&	autocomplete the current command or path \\
	Ctrl+I	&	same as Tab	\\
	Alt+*	&	insert all possible completions	\\
	\important{Ctrl+C}	&	kill the current command	\\
	\important{Ctrl+D}	&	exit the current shell (write end-of-file character)	\\
	Ctrl+X Ctrl+E	&	write the next command in your \$EDITOR	\\
	\important{Ctrl+R}	& reverse-search your history for a command	\\
	Ctrl+Z	&	suspend the process. Resume with \code{\%}	\\
	\attention{Ctrl+S}	&	suspend the current terminal	\\
	\important{Ctrl+Q}	&	resume a suspended terminal	\\
	Ctrl+L	&	clear the terminal. Similar to \code{clear}	\\
	Ctrl+U	&	clear the line before the cursor	\\
	Ctrl+K	&	clear the line after the cursor	\\
	Alt+F	&	move forward one word	\\
	Alt+B	&	move backward one word	\\
	Alt+D	&	delete next word	\\
	Alt+Backspace	&	delete previous word	\\
}

\section{Redirecting Standard I/O}
\mytable{gX}{
	\important{\code{\cmd{} > \file{}}}	&	write output to a new \file{} (overwrites)	\\
	\important{\code{\cmd{} >> \file{}}}	&	append output to \file{}	\\
	\code{\cmd{} | tee \file{}}	&	both print and write to a file (add \code{-a} to append)	\\
	\code{\cmd{} 2> \file{}}	&	write errors to \file{}	\\
	\code{\cmd{} 2>\&1}	&	redirect errors to standard output	\\
	\important{\code{\cmd{} \&>/dev/null}}	&	discard all output	\\
	\code{\cmd{} < \file{}}	&	read input from \file{}	\\
	\code{\cmd{} << EOF}	&	read input from command line until the line "\code{EOF}"	\\
	\code{\cmd{} <<< \cmd{}}	&	read input from the rest of the line	\\
	\code{\cmd{} | \cmd{}}	&	pipe output from the first \cmd{} to the second	\\
}

\section{Process Control (Forking and Killing)}
\mytable{gX}{
	\important{\code{\cmd{} \&}}	&	Send \cmd{} to background, return to command line	\\
	\code{wait}	&	wait for forked processes to finish	\\
	\code{( \cmd{} \& );exit}	&	fork a command within a one-liner (example)x	\\
	\attention{\code{killall \cmd{}}}	&	stop all processes with the name \cmd{}	\\
	\attention{\code{kill \pid{}}}	&	ask a process to stop	\\
	\attention{\code{kill -KILL \pid{}}}	&	forcefully stop a process	\\
}

\section{Automatic String Expansion (Examples)}
\mytable{gX}{
	\important{\code{echo *.txt}}	&	asd.txt dsa.txt longfilename.txt s.txt	\\
	\important{\code{echo ?s?.t?t}}	&	asd.txt dsa.txt bse.tot	\\
	\code{echo \{7..11\}}	&	7 8 9 10 11	\\
	\code{echo \{07..11\}}	&	07 08 09 10 11	\\
	\code{echo \{a..g\}}	&	a b c d e f g	\\
	\code{echo sim\{08..10\}}	&	sim08 sim09 sim10	\\
	\code{echo foo.\{txt,pdf,png\}}	&	foo.txt foo.pdf foo.png	\\
}

\section{Flow Control}
\mytable{gX}{
	\important{\specialcell{\code{if \textit{expression}; then} \\ \code{  \textit{do something}} \\ \code{else} \\ \code{  \textit{do something else}}\\ \code{fi}}}	&	Expressions can be commands and functions (return 0 $\rightarrow$ true) or built-in conditionals	\\
	\code{\textit{expression} \&\& \cmd{}}	&	run \cmd{} if \textit{expression} is true	\\
	\code{\textit{expression} || \cmd{}}	&	run \cmd{} if \textit{expression} is false	\\
}

\section{Aborting and Exiting}
\mytable{gXgX}{
	\code{continue}	&	next loop iteration	&
	\code{break}	&	exit loop	\\
	\code{return}	&	exit function	&
	\code{exit}	&	exit script / terminal	\\
}

\section{Unary Conditionals}
\mytable{gXgX}{
	\important{\code{[ -z \str{} ]}}	&	\str{} is empty		&	\important{\code{[ -n \str{} ]}}	&	\str{} is not empty	\\
	\code{[ -e \file{} ]}	&	\file{} exists	&	\code{[ -s \file{} ]}	&	\file{} is not empty	\\
	\code{[ -f \file{} ]}	&	\file{} is a regular file	&	\code{[ -d \dir{} ]}	&	dir{} is a directory	\\
	\code{[ -L \file{} ]}	&	\file{} is a symlink	&	\code{[ -x \file{} ]}	&	\file{} is executable	\\
	\code{[ -r \file{} ]}	&	\file{} is readable	&	\code{[ -w \file{} ]}	&	\file{} is writable	\\
	\code{[ -v \str{} ]}	&	\str{} is a variable	&	\code{[ -O \file{} ]}	&	\$USER owns \file{} 	\\
}

\section{Binary Conditionals}
\mytable{gXgX}{
	\important{\specialcell{\code{[ \textit{arg1} < \textit{arg2} ]}\\\code{[ \textit{arg1} > \textit{arg2} ]}\\\code{[ \textit{arg1} == \textit{arg2} ]}\\\code{[ \textit{arg1} != \textit{arg2} ]}}}	&	strings	&
	\specialcell{\code{[[ \textit{arg1} < \textit{arg2} ]]}\\\code{[[ \textit{arg1} > \textit{arg2} ]]}\\\code{[[ \textit{arg1} == \textit{arg2} ]]}\\\code{[[ \textit{arg1} != \textit{arg2} ]]}}	&	\specialcell{raw strings\\(no string\\expansion)}	\\
	\specialcell{\code{[ \textit{arg1} -lt \textit{arg2} ]}\\\code{[ \textit{arg1} -gt \textit{arg2} ]}\\\code{[ \textit{arg1} -eq \textit{arg2} ]}\\\code{[ \textit{arg1} -ne \textit{arg2} ]}}	&	integers	&
	\specialcell{\code{(( \textit{arg1} < \textit{arg2} ))}\\\code{(( \textit{arg1} > \textit{arg2} ))}\\\code{(( \textit{arg1} == \textit{arg2} ))}\\\code{(( \textit{arg1} != \textit{arg2} ))}}	&	integers	\\
}

\section{Loops}
\mytable{gX}{
	\important{\specialcell{\code{for word in \$words; do}\\ \code{ echo \$word}\\ \code{done}}}	&	print every word in \$words\\
	\specialcell{\code{while \textit{expression}; do}\\\code{  \textit{do something}}\\\code{done}}	&	traditional while loop	\\
	\multicolumn{2}{|c|}{Some ways of iterating over integers}	\\
	\multicolumn{2}{|i|}{\code{for i in \{0..9\}; do echo \$i; done}}	\\
	\multicolumn{2}{|g|}{\code{for i in `seq \$start \$num \$step`; do echo \$i; done}}	\\
	\multicolumn{2}{|a|}{\code{i=0; while (( num < 10 )); do echo \$i; let i++; done}}	\\
	\multicolumn{2}{|c|}{iterate over every line in \$var:}	\\
	\multicolumn{2}{|g|}{\code{IFS=\$'\textbackslash r\textbackslash n' ; for line in \$var; do echo \$line; done}}	\\
}

\section{Parallel Workers}
\mytable{gX}{
	\specialcell{
		\code{num=100}	\\
		\code{next()\{}	\\
		\code{  (( num > 0 )) \&\& let num-- || return}	\\
		\code{  \cmd{}\&}	\\
		\code{\}}	\\
		\code{set -o monitor}	\\
		\code{trap next CHLD}	\\
		\code{for i in `grep proc /proc/cpuinfo`;do}	\\
		\code{  next}	\\
		\code{done}	\\
		\code{wait}	\\
		\code{trap - CHLD}	\\
	}	&	Run \cmd{} 100 times in a parallel queue, one process per cpu core	\\
}

\section{I/O Processing}
\mytable{gX}{
	\important{\code{\cmd{} \$@}}	&	process all arguments at once	\\
	\specialcell{
		\code{while true; do}	\\
		\code{  \cmd{} "\$1"}	\\
		\code{  shift || break}	\\
		\code{done}	\\
	}	&	process arguments separately. To be used in a script or function.	\\
	\important{\code{\cmd{} | xargs}}	&	merge output to a single line	\\
	\important{\code{echo "foo bar" | xargs \cmd{}}}	&	set arguments of \cmd{} to \code{foo bar}	\\
	\important{\code{echo "foo bar" | xargs -n1}}	&	split to one word per line	\\
	\important{\code{echo "foo bar" | xargs -n1 \cmd{}}}	&	run \cmd{} on every single word	\\
	\code{read myvar}	&	read a line from stdin into \$myvar	\\
}

\section{Bash Invocation}
\mytable{gX}{
	\code{bash -c "\cmd{}"}	&	run \cmd{} in a fresh bash instance	\\
	\code{su \user{} -c "\cmd{}"}	&	run \cmd{} as another user \\
	\attention{\code{sudo "\cmd{}"}}	&	run \cmd{} as root \\
	\code{ssh \user{}@\host{} \cmd{}}	&	run \cmd{} as \user{} on \host{}	\\
}


%\section{Further Reading}
%
%\begin{tabularx}{0.33\textwidth}{X}
%\textbf{BASH Programming - Introduction HOW-TO}\\
%http://www.faqs.org/docs/Linux-HOWTO/Bash-Prog-Intro-HOWTO.html
%\\
%\textbf{Bash Manpage (online for faster searching)}\\
%http://linux.die.net/man/1/bash
%\end{tabularx}

\end{multicols*}

\end{document}
