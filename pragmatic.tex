\documentclass[7pt]{scrartcl}
\usepackage{multicol}
\usepackage{calc}
\usepackage{ifthen}
\usepackage[landscape]{geometry}
\usepackage{xcolor}
\usepackage{colortbl}
\usepackage{tabularx}
\usepackage{multirow}
\usepackage{graphicx}

\ifthenelse{\lengthtest { \paperwidth = 11in}}
	{ \geometry{top=.5in,left=.5in,right=.5in,bottom=.5in} }
	{\ifthenelse{ \lengthtest{ \paperwidth = 297mm}}
		{\geometry{top=1cm,left=1cm,right=1cm,bottom=1cm} }
		{\geometry{top=1cm,left=1cm,right=1cm,bottom=1cm} }
	}
	
\pagestyle{empty}

\makeatletter
\renewcommand{\section}{\@startsection{section}{1}{0mm}%
                                {-2ex plus -.5ex minus -.2ex}%
                                {0.5ex plus .2ex}%x
                                {\normalfont\large\bfseries}}
\makeatother

\def\BibTeX{{\rm B\kern-.05em{\sc i\kern-.025em b}\kern-.08em
    T\kern-.1667em\lower.7ex\hbox{E}\kern-.125emX}}

% Don't print section numbers
\setcounter{secnumdepth}{0}


\setlength{\parindent}{0pt}
\setlength{\parskip}{0pt plus 0.5ex}

\setlength{\premulticols}{0pt}
\setlength{\postmulticols}{0pt}
\setlength{\multicolsep}{0pt}
\setlength{\columnsep}{0pt}

\raggedright
\footnotesize

\definecolor{Gray}{gray}{0.85}
\definecolor{Red}{rgb}{1.0 0.7 0.7}
\definecolor{Green}{rgb}{0.7 1.0 0.7}
\newcolumntype{g}{>{\columncolor{Gray}}l}
\newcolumntype{G}{>{\columncolor{Gray}}X}
\newcolumntype{a}{>{\columncolor{Red}}l}
\newcolumntype{A}{>{\columncolor{Red}}X}
\newcolumntype{i}{>{\columncolor{Green}}l}
\newcolumntype{I}{>{\columncolor{Green}}X}

\newcommand{\mytable}[2]{\begin{tabularx}{0.33\textwidth}{|#1|}\hline #2 \hline\end{tabularx}}
\newcommand{\specialcell}[2][l]{\begin{tabular}[#1]{@{}l@{}}#2\end{tabular}}
\newcommand{\sct}[1]{\specialcell[t]{#1}}
%\newcommand*{\thead}[1]{\multicolumn{1}{c}{\bfseries #1}}
\newcommand{\thead}[1]{\textbf{#1}}

\newcommand{\myopt}[1]{\hspace{3ex}\myoptc{#1}}
\newcommand{\myoptc}[1]{\texttt{#1}}
\newcommand{\myoptd}[1]{\hspace{3ex}\fontsize{6pt}{6pt}\selectfont #1}
\newcommand{\code}[1]{\texttt{\textbf{#1}}}

\newcommand{\attention}[1]{\multicolumn{1}{|a}{#1}}
\newcommand{\important}[1]{\multicolumn{1}{|i}{#1}}

\newcommand{\repo}[0]{\textit{repo}}
\newcommand{\commit}[0]{\textit{commit}}
\newcommand{\branch}[0]{\textit{branch}}
\newcommand{\rbranch}[0]{\textit{rbranch}}
\newcommand{\file}[0]{\textit{file}}
\newcommand{\dir}[0]{\textit{dir}}
\newcommand{\tag}[0]{\textit{tag}}
\newcommand{\remote}[0]{\textit{remote}}
\newcommand{\url}[0]{\textit{url}}

\begin{document}

\setlength{\columnsep}{1em}

\begin{multicols*}{5}
\begin{center}
\Large{\textbf{The Pragmatic Programmer\\Quick Reference Guide}}\\
\large{as included in the book\\\today}
\end{center}

\tipitem{Care About Your Craft}{
Why spend your life developing software unless you care about doing it well?
}

\tipitem{Think! About Your Work}{
Turn off the autopilot and take control. Constantly critique and appraise your work.
}
 
\tipitem{Provide Options, Don't Make Lame Excuses}{
Instead of excuses, provide options. Don't say it can't be done; explain what can be done.
}

\tipitem{Don't Live with Broken Windows}{
Fix bad designs, wrong decisions, and poor code when you see them.
}
 
\tipitem{Be a Catalyst for Change}{
You can't force change on people. Instead, show them how the future might be and help them participate in creating it.
}
 
\tipitem{Remember the Big Picture}{
Don't get so engrossed in the details that you forget to check what's happening around you.
}
 
\tipitem{Make Quality a Requirements Issue}{
Involve your users in determining the project's real quality requirements.
}
 
\tipitem{Critically Analyze What You Read and Hear}{
Don't be swayed by vendors, media hype, or dogma. Analyze information in terms of you and your project.
}
 
\tipitem{Invest Regularly in Your Knowledge Portfolio}{
Make learning a habit.
}
 
\tipitem{It's Both What You Say and the Way You Say It}{
There's no point in having great ideas if you don't communicate them effectively.
}
 
\tipitem{DRY - Don't Repeat Yourself}{
Every piece of knowledge must have a single, unambiguous, authoritative representation within a system.
}
 
\tipitem{Make It Easy to Reuse}{
If it's easy to reuse, people will. Create an environment that supports reuse.
}
 
\tipitem{Eliminate Effects Between Unrelated Things}{
Design components that are self-contained, independent, and have a single, well-defined purpose.
}
 
\tipitem{There Are No Final Decisions}{
No decision is cast in stone. Instead, consider each as being written in the sand at the beach, and plan for change.
}
 
\tipitem{Use Tracer Bullets to Find the Target}{
Tracer bullets let you home in on your target by trying things and seeing how close they land.
}
 
\tipitem{Prototype to Learn}{
Prototyping is a learning experience. Its value lies not in the code you produce, but in the lessons you learn.
}
 
\tipitem{Program Close to the Problem Domain}{
Design and code in your user's language.
}
 
\tipitem{Estimate to Avoid Surprises}{
Estimate before you start. You'll spot potential problems up front.
}
 
\tipitem{Iterate the Schedule with the Code}{
Use experience you gain as you implement to refine the project time scales.
}
 
\tipitem{Keep Knowledge in Plain Text}{
Plain text won't become obsolete. It helps leverage your work and simplifies debugging and testing.
}
 
\tipitem{Use the Power of Command Shells}{
Use the shell when graphical user interfaces don't cut it.
}
 
\tipitem{Use a Single Editor Well}{
The editor should be an extension of your hand; make sure your editor is configurable, extensible, and programmable.
}
 
\tipitem{Always Use Source Code Control}{
Source code control is a time machine for your work - you can go back.
}
 
\tipitem{Fix the Problem, Not the Blame}{
It doesn't really matter whether the bug is your fault or someone else's - it is still your problem, and it still needs to be fixed.
}
 
\tipitem{Don't Panic When Debugging}{
Take a deep breath and THINK! about what could be causing the bug.
}
 
\tipitem{``select'' Isn't Broken}{
It is rare to find a bug in the OS or the compiler, or even a third-party product or library. The bug is most likely in the application.
}
 
\tipitem{Don't Assume It - Prove It}{
Prove your assumptions in the actual environment - with real data and boundary conditions.
}
 
\tipitem{Learn a Text Manipulation Language}{
You spend a large part of each day working with text. Why not have the computer do some of it for you?
}
 
\tipitem{Write Code That Writes Code}{
Code generators increase your productivity and help avoid duplication.
}
 
\tipitem{You Can't Write Perfect Software}{
Software can't be perfect. Protect your code and users from the inevitable errors.
}
 
\tipitem{Design with Contracts}{
Use contracts to document and verify that code does no more and no less than it claims to do.
}
 
\tipitem{Crash Early}{
A dead program normally does a lot less damage than a crippled one.
}
 
\tipitem{Use Assertions to Prevent the Impossible}{
Assertions validate your assumptions. Use them to protect your code from an uncertain world.
}
 
\tipitem{Use Exceptions for Exceptional Problems}{
Exceptions can suffer from all the readability and maintainability problems of classic spaghetti code. Reserve exceptions for exceptional things.
}
 
\tipitem{Finish What You Start}{
Where possible, the routine or object that allocates a resource should be responsible for deallocating it.
}
 
\tipitem{Minimize Coupling Between Modules}{
Avoid coupling by writing “shy” code and applying the Law of Demeter.
}
 
\tipitem{Configure, Don't Integrate}{
Implement technology choices for an application as configuration options, not through integration or engineering.
}
 
\tipitem{Put Abstractions in Code, Details in Metadata}{
Program for the general case, and put the specifics outside the compiled code base.
}
 
\tipitem{Analyze Workflow to Improve Concurrency}{
Exploit concurrency in your user's workflow.
}
 
\tipitem{Design Using Services}{
Design in terms of services - independent, concurrent objects behind well-defined, consistent interfaces.
}
 
\tipitem{Always Design for Concurrency}{
Allow for concurrency, and you'll design cleaner interfaces with fewer assumptions.
}
 
\tipitem{Separate Views from Models}{
Gain flexibility at low cost by designing your application in terms of models and views.
}
 
\tipitem{Use Blackboards to Coordinate Workflow}{
Use blackboards to coordinate disparate facts and agents, while maintaining independence and isolation among participants.
}
 
\tipitem{Don't Program by Coincidence}{
Rely only on reliable things. Beware of accidental complexity, and don't confuse a happy coincidence with a purposeful plan.
}
 
\tipitem{Estimate the Order of Your Algorithms}{
Get a feel for how long things are likely to take before you write code.
}
 
\tipitem{Test Your Estimates}{
Mathematical analysis of algorithms doesn't tell you everything. Try timing your code in its target environment.
}
 
\tipitem{Refactor Early, Refactor Often}{
Just as you might weed and rearrange a garden, rewrite, rework, and re-architect code when it needs it. Fix the root of the problem.
}
 
\tipitem{Design to Test}{
Start thinking about testing before you write a line of code.
}
 
\tipitem{Test Your Software, or Your Users Will}{
Test ruthlessly. Don't make your users find bugs for you.
}
 
\tipitem{Don't Use Wizard Code You Don't Understand}{
Wizards can generate reams of code. Make sure you understand all of it before you incorporate it into your project.
}
 
\tipitem{Don't Gather Requirements - Dig for Them}{
Requirements rarely lie on the surface. They're buried deep beneath layers of assumptions, misconceptions, and politics.
}
 
\tipitem{Work with a User to Think Like a User}{
It's the best way to gain insight into how the system will really be used.
}
 
\tipitem{Abstractions Live Longer than Details}{
Invest in the abstraction, not the implementation. Abstractions can survive the barrage of changes from different implementations and new technologies.
}
 
\tipitem{Use a Project Glossary}{
Create and maintain a single source of all the specific terms and vocabulary for a project.
}
 
\tipitem{Don't Think Outside the Box - Find the Box}{
When faced with an impossible problem, identify the real constraints. Ask yourself: “Does it have to be done this way? Does it have to be done at all?”
}
 
\tipitem{Start When You're Ready}{
You've been building experience all your life. Don't ignore niggling doubts.
}
 
\tipitem{Some Things Are Better Done than Described}{
Don't fall into the specification spiral - at some point you need to start coding.
}
 
\tipitem{Don't Be a Slave to Formal Methods}{
Don't blindly adopt any technique without putting it into the context of your development practices and capabilities.
}
 
\tipitem{Costly Tools Don't Produce Better Designs}{
Beware of vendor hype, industry dogma, and the aura of the price tag. Judge tools on their merits.
}
 
\tipitem{Organize Teams Around Functionality}{
Don't separate designers from coders, testers from data modelers. Build teams the way you build code.
}
 
\tipitem{Don't Use Manual Procedures}{
A shell script or batch file will execute the same instructions, in the same order, time after time.
}
 
\tipitem{Test Early. Test Often. Test Automatically.}{
Tests that run with every build are much more effective than test plans that sit on a shelf.
}
 
\tipitem{Coding Ain't Done ‘Til All the Tests Run}{
‘Nuff said.
}
 
\tipitem{Use Saboteurs to Test Your Testing}{
Introduce bugs on purpose in a separate copy of the source to verify that testing will catch them.
}
 
\tipitem{Test State Coverage, Not Code Coverage}{
Identify and test significant program states. Just testing lines of code isn't enough.
}
 
\tipitem{Find Bugs Once}{
Once a human tester finds a bug, it should be the last time a human tester finds that bug. Automatic tests should check for it from then on.
}
 
\tipitem{English is Just a Programming Language}{
Write documents as you would write code: honor the DRY principle, use metadata, MVC, automatic generation, and so on.
}
 
\tipitem{Build Documentation In, Don't Bolt It On}{
Documentation created separately from code is less likely to be correct and up to date.
}
 
\tipitem{Gently Exceed Your Users' Expectations}{
Come to understand your users' expectations, then deliver just that little bit more.
}
 
\tipitem{Sign Your Work}{
Craftsmen of an earlier age were proud to sign their work. You should be, too.
}

\hfill \textbf{The Pragmatic Programmer}\\
\hfill by Andrew Hunt and David Thomas\\
\hfill ISBN: 0-201-61622-X
 
\end{multicols*}

\clearpage

\end{document}
